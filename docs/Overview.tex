\documentclass[letterpaper,10pt]{article}

\usepackage{fullpage}

\usepackage{amsmath}
\usepackage{amssymb}
\usepackage{amsthm}
\usepackage{nameref}
\usepackage{url}

\usepackage[underline=true,rounded corners=false]{pgf-umlsd}

\title{CSAL Database Project Introduction}
\author{Craig Kelly}

% Macro for section num AND section name references
\newcommand{\fullxref}[1]{ \ref{#1} \nameref{#1} }

\begin{document}

% Force pdflatex to properly use letter as page size (instead
% of defaulting to A4)
\special{papersize=8.5in,11in}
\setlength{\pdfpageheight}{\paperheight}
\setlength{\pdfpagewidth}{\paperwidth}

\setlength{\parindent}{0pt}
\setlength{\parskip}{6pt}

\maketitle
\tableofcontents
\pagebreak

%%%%%%%%%%%%%%%%%%%%%%%%%%%%%%%%%%%%%%%%%%%%%%%%%%%%%%%%%%%%%%%%%%%%%%%%%%%%
%%%%%%%%%%%%%%%%%%%%%%%%%%%%%%%%%%%%%%%%%%%%%%%%%%%%%%%%%%%%%%%%%%%%%%%%%%%%

\section{Introduction}

This repository contains code for storing and displaying data stored as
part of the CSAL project.  The use cases and the code produced are described
below.  The short version is that the data is stored in a MongoDB instance,
there is a C\# library for accessing the database, and there is a Web API
wrapping the DLL.  In addition, the Web API server provides a very simple
ASP.NET MVC application for viewing the data.  Because of the dependence
of the Web API on the ``core'' library, the only deployment information is below
in \fullxref{subsec:deploy}

%%%%%%%%%%%%%%%%%%%%%%%%%%%%%%%%%%%%%%%%%%%%%%%%%%%%%%%%%%%%%%%%%%%%%%%%%%%%

\section{Use Cases}

\subsection{Direct Logging}
\label{use:directwrite}

Services running on the same server as the database (i.e. ACE) want the
ability to write JSON data log entries without POST'ing to an HTTP endpoint.
Note that while the architecture of the project allows this use case (and it
is currently used in production), a safe alternative would be to force all
applications to write via the ReST API (see \fullxref{use:restwrite}).

\subsection{Logging via ReST Endpoint}
\label{use:restwrite}

Applications may post a JSON record representing an ACE turn to a public
endpoint for persisting in the database. Note that this is the recommended
way to save data to the database (contrast with \fullxref{use:directwrite})

\subsection{Teacher Status Check}

Teachers need to be able to see students' progress in the lesson.  There
should be a way to see how the entire class is doing, how the class is doing
on a lesson, how a student is doing across the lessons, and how a specified
student is doing on a specific lesson.  Note that if this application grows,
this Use Case should be broken out into small chunks

\subsection{Data Administration}

There must be a way for administrators (not teachers) to initialize and edit
the database.  Specifically, class, lesson, and student data should be
configured for expected logging (via either use case
\fullxref{use:directwrite} or \fullxref{use:restwrite}).

%%%%%%%%%%%%%%%%%%%%%%%%%%%%%%%%%%%%%%%%%%%%%%%%%%%%%%%%%%%%%%%%%%%%%%%%%%%%

\section{A Word About Dates}

Any system logging or reporting events that occur in time must deal with the
how time is represented.  It is important to remember that this application
will generally deal with dates with the standard .NET libraries, but the
actual dates are stored in a MonogDB database.  In addition, for some tasks
it is better to have a quick way of determining event order and spacing;
as a result, some dates are stored as a number of milliseconds since a
specified ``epoch''.

\subsection{General Date Handling}

Generally speaking, dates are handled as instances of .NET DateTime objects.
Local time and formats are used, so as of now dates are displayed using the
US standard and in Central Standard Time (``Memphis'' or ``Chicago'' time).

\textbf{However}, MongoDB prefers to store dates in UTC (or ``Zulu'' time).
As a result, applications using a DateTime instance that has been read back
from the database will need to convert the DateTime instance via a call to 
\texttt{DateTime.ToLocalTime}. Currently this only applies to DateTime-valued
properties on classes in the namespace \texttt{CSALMongo.Model}.

\subsection{Epoch-Based Dates}

There are two float-valued time-related fields in a Turn as stored in the 
Student Acts collection in the database (see \fullxref{db:turns}).

The first is ``duration''.  It comes from the logging application and
represents the length of the turn being logged in milliseconds.

The second is ``DBTimestamp''.  It is generated by the logging acceptance code
in our class \texttt{CSALMongo.CSALDatabase}.  It is calculated as the number
of milliseconds since January 1\textsuperscript{st} of the year
\texttt{CSALMongo.TurnModel.ConvLog.EPOCH\_YR}, which is currently 2010.

If ``DBTimestamp'' is found in the JSON data passed in (as if the
logging application generated it), it is treated as the current time.  This
allows previously saved turns to be ``re-posted'' to the database via the ReST
API.  For instance, the script scripts/turn\_copy.py in this project relies
on this functionality.

\subsection{Date Correction}

Some instances of ``DBTimestamp'' might be flawed; for instance, records
collected before this functionality were added have spurious, too-close
timestamps.  To attempt to keep the time intervals for the test system
(or a production system with some kind of issues) displayed in a sane manner, 
we use the method \texttt{CSALMongo.Model.StudentLessonActs.FixupTimestamps}.
This method insures that the timestamp of any turn $n$ is a ``minimally pretty
valid'' timestamp.  

Every turn's timestamp is fixed so that
$ ts_{n} \geq ts_{n-1} + dur_{n-1} + 200 $ where ``ts'' is a timestamp and
``dur'' is a duration. This insures that a Turn won't appear to happen during
the previous turn.  Also note the arbitrarily chosen extra margin of 200
milliseconds.

%%%%%%%%%%%%%%%%%%%%%%%%%%%%%%%%%%%%%%%%%%%%%%%%%%%%%%%%%%%%%%%%%%%%%%%%%%%%

\section{Database}
\label{sec:database}

The main documentation for the JSON logging record is available in a Google
doc.  Please see the documents ``CSAL Data'' at 
\url{https://docs.google.com/document/d/19nJZMReWpTat_tjNeOvA7oa8rDD0KmQ5oKaHnn2HmwE}
and ``AutoTutor Conversation Engine (ACE) Web API (CORS Version)'' at
\url{https://docs.google.com/document/d/1ZRlj7e5u4PQSlggCD--yZ5EsB2KCZzt7P8MEI6HB9So}

The various database entities are described below.  The JSON logging data
described above is stored in the entity described in \fullxref{db:turns}. You
may also see how the C\# classes for this data (both the JSON logging format
and the database entities below) are structured by looking in the CSALMongo
project or the CSALMongo.chm compiled documentation in this directory.

The server is autotutor.x-in-y.com and the MongoDB database should be named
csaldata.  There are four collections: classes, lessons, students, and
studentActions.

\subsection{Class}
\label{db:class}

There is one document in this collection per class.  It contains a list of the
students in the class and the lessons used. Please see \fullxref{db:turns} for
details on auto-creation and updating.

\subsection{Student}
\label{db:student}

There is one document in this collection per student. Please see
\fullxref{db:turns} for details on auto-creation and updating.

\subsection{Lesson}
\label{db:lesson}

There is one document in this collection per lesson. Please see
\fullxref{db:turns} for details on auto-creation and updating.

\subsection{Student Actions}
\label{db:turns}

There is one document in this collection per student per lesson. Any time a
JSON logging record is saved for a student in a lesson, it is appended to
the Turns list in the corresponding document in this collection. Although it
is preferred to have the class, lesson, and student documents matching this
information pre-populated, a minimal version of each entity will be created
if it is not present when the data is logged.  Auto-created entities will have
a property named AutoCreated set to true.

To help with queries, we also update the class, student, and lesson documents
when we receive turn data like so:

\begin{itemize}
    \item \fullxref{db:class} - update the fields Students and Lessons
    \item \fullxref{db:lesson} -  update the fields LastTurnTime, Students,
          AttemptTimes, StudentsAttempted, StudentsCompleted, and URLs
    \item \fullxref{db:student} -  update the fields LastTurnTime and TurnCount
\end{itemize}



%%%%%%%%%%%%%%%%%%%%%%%%%%%%%%%%%%%%%%%%%%%%%%%%%%%%%%%%%%%%%%%%%%%%%%%%%%%%

\section{CSALMongo DLL}

The ``base'' or ``core'' DLL contains the model classes for JSON logging
format, the model classes for the database, the actual database interface
class, and some supporting code.  The project is documented via XML documentation
which has been compiled into the CSALMongo.chm in this directory.

There is also a unit test project named CSALMongoUnitTest. It uses the Unit
Testing facilities availble with Visual Studio 2013. The tests are broken into
five categories of testing:

\begin{enumerate}
    \item Model Testing for methods added the model classes for
          parsing, information, etc.
    
    \item Database Operations Testing for actual database operations exposed
          by the main class
    
    \item Database Utility Testing for helper or utility methods exposed by
          the main database class
    
    \item Utility Testing for helper or utility functions \textbf{outside} the 
          main database class.
\end{enumerate}



%%%%%%%%%%%%%%%%%%%%%%%%%%%%%%%%%%%%%%%%%%%%%%%%%%%%%%%%%%%%%%%%%%%%%%%%%%%%

\section{CSALMongo Web API}
\label{sec:webapi}

\subsection{ReST API}

The ReST API is exposed via a .NET Web Api project (that also houses a GUI - 
see \fullxref{web:gui}).

Since this is a ReST API, there is a URL namespace complete with expected verbs
and payloads.  They are documented below. You may also see the Python script
\url{../scripts/db_init.py} for an example
of using ReST API.  Essentially, the rules are:

\begin{itemize}
    \item Lesson, Class, and Student ID's placed in a URL should be double-escaped
          (see also RenderHelp in the Web API project)
    \item When POST'ing, set the header ``Content-Type'' to ``application/json''
\end{itemize}

Here are the API endpoints. It should be assumed that for local
workstation testing the url would begin with \url{http://localhost:62702}.  For
the production URL, the proper prefix would be \url{http://autotutor.x-in-y.com/csaldb}.

\begin{itemize}
    \item GET /api/classes - returns all Classes in the database
    \item GET /api/classes/\$id (where \$id is ClassID) - returns the specified Class
    \item POST /api/classes/\$id (where \$id is ClassID) - the posted body should be
          a JSON document matching Model.Class.
          
    \item GET /api/lessons - returns all Lessons in the database
    \item GET /api/lessons/\$id (where \$id is LessonID) - return the specified Lesson
    \item POST /api/lessons/\$id (where \$id is LessonID) - the posted body should be
          a JSON document matching Model.Lesson.

    \item GET /api/studentreading/\$id (where \$id is StudentID) - return list of
          reading URL's and timestamps (in order they were POST'ed)
    \item POST /api/studentreading - the posted body should be a JSON document with
          two fields: UserID and TargetURL.  UserID is a StudentID.

    \item GET /api/studentsatlocation/\$location (where location is a string matching
          the location field of one or more classes).  Return a list of students (JSON
          formatted to a match Model.Student) that are in classes matching the specified
          location.

    \item GET /api/students - returns all Students in the database
    \item GET /api/students/\$id (where \$id is StudentID) - return the specified Student
    \item POST /api/students/\$id (where \$id is StudentID) - the posted body should be
          a JSON document matching Model.Student.
    
    \item GET /api/turn/\$lesson/\$user (where \$lesson is a LessonID and \$user is a
          Student/Subject ID) - return a list of logged turns in their original JSON
          format
    \item POST /api/turn - the posted JSON body should match the format as described
          in the documents linked in \fullxref{sec:database} or in the TurnModel.ConvLog.
    
    \item GET /api/turnrollup/\$lesson/\$user (where \$lesson is a LessonID and \$user is a
          Student/Subject ID) - return a ``rolled up'' list of turns.  Each item in the list
          represents an attempt as logged as part of the JSON API.
    
    \item GET/POST /api/maker - for testing and should probably not be used at this point
\end{itemize}

\subsection{User Interface}
\label{web:gui}
The User Interface is an ASP.NET MVC web application, served by the Home controller
class, rendered via Razor.  It uses jQuery and Bootstrap for UI automation
and styling.  Various jQuery UI plugins are also used (notably DataTables and
Sparklines).

The application maintains a URL namespace similar to the ReST API, but under
the Home directory.  As above, It should be assumed that for local
workstation testing the url would begin with \url{http://localhost:62702}.  For
the production URL, the proper prefix would be 
\url{http://autotutor.x-in-y.com/csaldb}.

Note that if a user access one of these endpoints (with the exception of those
related to authentication) and is not logged in, a login/redirection will occur.
In addition, a users' views will be filtered by classes that their login email
matches (in the TeacherName field).  The one exception is administrators. Please
see \fullxref{web:auth} for details.


\begin{itemize}
    \item /home/logout - clears the current user session and redirects to the index page
    \item /home/login - starts the Google OAuth2 login process
    \item /home/oauth2callback - the endpoint the Google OAuth2 servers will use for
          redirection after user authentication
    
    \item /home/index - the default/index page
    
    \item /home/testing - shows the silly testing page, which should probably be removed
    
    \item /home/classes - shows the list of classes viewable by the current user
    
    \item /home/classdetails/\$id (where \$id is a ClassID) - display details for the
          indicated class
    
    \item /home/studentlessondrill/\$lesson/\$user (where \$lesson is a LessonID and \$user is a
          Student/Subject ID).  Construct and display a drill-down into a student's work on
          a single lesson
    
    \item /home/studentlessondevselect (for administrators only) - allow the user to select 
          a student/lesson combination for viewing a debug drill-down
    
    \item /home/studentlessondevview/\$lesson/\$user (where \$lesson is a LessonID and \$user is a
          Student/Subject ID - for administrators only).  Construct and display a debug
          drill-down into a student's work on a single lesson
    
    \item /home/lessons - shows the list of lessons viewable by the current user
    
    \item /home/lessondetails/\$id (where \$id is a LessonID) - display details for the
          indicated lesson
          
    \item /home/students - shows the list of students viewable by the current user
    
    \item /home/studentdetails/\$id (where \$id is a StudentID) - display details for the
          indicated student
    
    \item /home/materials - view the current hard-coded list of teaching materials
\end{itemize}

\subsection{Authentication}
\label{web:auth}
Authentication is handled via Google OAuth2.  Administrators are identified
by email address in the web.config file.  Teachers are given access to class
information if the address from their OAuth2 profile matches the teacher name
for the class.

\subsection{Logging}

This document generally refers to logging to identify ACE turn records
sent in JSON format and stored in the MongoDB collection studentActions.
However, the Web API and GUI must log records as well.  Currently this is
fairly simple; in addition to default IIS logging, the Elmah library is
used to log unhandled exceptions and any errors displayed via the custom
user error page.  The Elmah log is only stored in-memory (and so is transient)
and can be accessed at the main URL at
\url{http://autotutor-x-in-y.com/csaldb/elmah}.

\subsection{Deployment}
\label{subsec:deploy}

The entire application should be deployed via Visual Studio packaging and IIS
application import.  Please note that the application is already deployed to
its own App Pool on the production server, so a new deployment should be an
overwrite (not a new application).

Any deployment should also be accompanied by an annotated tag in the Git
repository.

%%%%%%%%%%%%%%%%%%%%%%%%%%%%%%%%%%%%%%%%%%%%%%%%%%%%%%%%%%%%%%%%%%%%%%%%%%%%

\section{Sequence Diagrams}

\subsection{Example of User Interaction}

\begin{sequencediagram}
    \newthread{web}{GUI User}
    \newinst[1]{api}{Web API}
    \newinst[1]{dll}{CSALMongo DLL}
    \newinst[1]{mongo}{MongoDB}
    
    \begin{call}{web}{View Lessons}{api}{render view}
        \begin{call}{api}{findLessons()}{dll}{return list}
            \begin{call}{dll}{find()}{mongo}{query response}
            \end{call}
        \end{call}
    \end{call}
\end{sequencediagram}

\begin{sequencediagram}
\end{sequencediagram}


%%%%%%%%%%%%%%%%%%%%%%%%%%%%%%%%%%%%%%%%%%%%%%%%%%%%%%%%%%%%%%%%%%%%%%%%%%%%
%%%%%%%%%%%%%%%%%%%%%%%%%%%%%%%%%%%%%%%%%%%%%%%%%%%%%%%%%%%%%%%%%%%%%%%%%%%%

\end{document}
